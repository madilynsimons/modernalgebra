\documentclass{article}
\title{Homework 1}
\author{Madilyn Simons}
\date{}
\setlength{\parindent}{0pt}
\pdfpagewidth 8.5in
\pdfpageheight 11in

\usepackage{amssymb}


\begin{document}
\maketitle

\begin{enumerate}

\item \textbf{a} We can prove the statement by induction.  First, let $n = 1$.
As such, $n^2 = 1$.  By the Division Algorithm, $n^2 = 8\cdot 0 + 1$.  Thus, the
statement holds for $n = 1$.  Next, assume the statement holds for $1, 3, 5, ..., n$.
  Since $n$ is odd, the next consecutive odd number is $n+2$.  Because $n^2$
  leaves a remainder of 1 when divided by 8, let $n^2 = 8\cdot k + 1$ for some integer $k$.
  As such, $(n+2)^2 = n^2 + 4\cdot n + 4 = 8k + 4n + 1$.
  Therefore, $(n+2)^2$ leaves a remainder of 1 when divided by 8.  By induction,
  for any odd number $n$, $n$ leaves a remainder of 1 when divided by 8.

\textbf{b} Let $n = 2^{a_0} \cdot  {p_1}^{a_1} \cdot  {p_2}^{a_2} \cdot  ... \cdot  {p_k}^{a_k}$
be the prime factorization of $n$ such that $p_i$ are distinct prime numbers for
all $i$ and there is no $i$ such that $p_i = 2$.
As such, $n^2 = 2^{2a_0} \cdot  {p_1}^{2a_1} \cdot  {p_2}^{2a_2} \cdot  ... \cdot  {p_k}^{2a_k}$.
Since $p_i$ is odd for all $i$ (since all $p_i$ are primes and do not equal 2),
${p_1}^{a_1} \cdot  {p_2}^{a_2} \cdot  ... \cdot  {p_k}^{a_k}$ is also odd.
By $1a$, ${p_1}^{2a_1} \cdot  {p_2}^{2a_2} \cdot  ... \cdot  {p_k}^{2a_k}$ leaves
a remainder of 1 when divided by 8.  Let ${p_1}^{2a_1} \cdot  {p_2}^{2a_2} \cdot  ... \cdot  {p_k}^{2a_k} = 8b + 1$
for some integer $b$.  Thus, $n^2 = 2^{2a_0} \cdot  (8b + 1) = 4^{a_0} \cdot  (8b + 1) = 8 \cdot  (4^{a_0}b) + 4^{a_0}.$
Since multiples of 4 can only leave a remainder of either 0 or 4 when divided by 8
($4(2k) = 8k$ and $4(2k + 1) = 8k + 4$ for all integers $k$)
and $n^2 = 8 \cdot  (4^{a_0}b) + 4^{a_0}$, $n^2$ can only leave a remainder or 0 or 4
when divided by 8.

\item If $3 \nmid n$, then, by the Division Algorithm, either $n = 3k + 1$ or $n = 3k + 2$ for some integer $k$.  First, assume $n = 3k + 1$.  Thus, $n^2 - 1 = (3k + 1)^2 - 1 = 3 \cdot  (3k^2 + 2k)$.  Therefore, $3 \vert (n - 1)^2$ for this case.  Next, assume $n = 3k + 2$.  Thus, $n^2 - 1 = (3k + 2)^2 - 1 = 3 \cdot  (3k^2 + 4k + 1)$.  Therefore, $3 \vert (n - 1)^2$ for all $n$ such that $3 \nmid n$.

\item Let $a = {p_1}^{x_1} \cdot  ... \cdot  {p_k}^{x_k}$ and $b = {p_1}^{y_1} \cdot  ... \cdot  {p_k}^{y_k}$ be the prime factorizations of $a$ and $b$ respectively such that $x_i \geq 0$ and $y_i \geq 0$ for all $i \leq k$.  Let $c = {q_1}^{z_1} \cdot  ... \cdot  {q_j}^{z_j}$ be the prime factorization of $c$ given there are no $m, n$ such that $p_m = q_n$.  This holds because $(a, c) = 1$ and $(b, c) = 1$, meaning $a$ and $c$ have no common prime factors and $b$ and $c$ have no common prime factors.  Thus, $ab = {p_1}^{x_1 + y_1} \cdot  ... \cdot  {p_k}^{x_k + y_k}$.  Therefore, $ab$ and $c$ also do not have any common prime factors.  This implies that $(ab, c) = 1$.

\item Either $(a,b) = 1$ or $(a, b) \neq 1$.  First, suppose $(a, b) = 1$.  Since $a$ and $b$ do not have any common prime factors, $c$ must be a multiple of $ab$ in order to be divisible by both $a$ and $b$.  Next suppose $(a, b) \neq 1$.  Let $a = {p_1}^{x_1} \cdot  ... \cdot  {p_k}^{x_k}$ and $b = {p_1}^{y_1} \cdot  ... \cdot  {p_k}^{y_k}$ be the prime factorizations of $a$ and $b$ respectively such that $x_i \geq 0$ and $y_i \geq 0$ for all $i \leq k$.  If $a$ and $b$ both divide $c$, then $m = {p_1}^{max(x_1, y_1)} \cdot  ... \cdot  {p_k}^{max(x_k, y_k)}$ is a factor of $c$.  If $(a,b) = {p_1}^{min(x_1, y_1)} \cdot  ... \cdot  {p_k}^{min(x_k, y_k)}$, then  $m \cdot  (a,b) = ab$.  Let $c = m\cdot x$ for some integer $x$.  Thus, $c = m\cdot x\cdot (a,b) = a\cdot b\cdot x$ is divisible by $ab$.

\item If $(a, 6) = 1$, then either $a = 6x + 1$ or $a = 6x - 1$ for some integer $x$
(since $(k, 6) > 1$ for all $2 \leq k \leq 4$).
Similarly, either $b = 6y + 1$ or $b = 6y - 1$.  Suppose $a = 6x + 1$.  Consequently, $a^2 = 36x^2 + 12x + 1 = 12x(3x + 1) + 1$.  Either $x$ is even or odd.  First, assume $x$ is even, so $x = 2k$ for some integer $k$.  Therefore, $a^2 = 24k(6k+1) + 1$, which leaves a remainder of 1 when divided by 24.  If $x$ is odd, then $x = 2k+1$ for some integer $k$.  As such, $a^2 = 24(6k^2 + 7k + 2) + 1$, which also leaves a remainder of 1 when divided by 24.  Next, suppose $a = 6x - 1$.  By the same logic as before, $a^2 = 24k(6k - 1) + 1$ when $x$ is even and $a^2 = 24(6k^2 + 5k +1) + 1$ when $x$ is odd, both of which leave a remainder of 1 when divided by 24.  Therefore, $a^2$ always leaves a remainder of 1 when divided by 24.  By the same logic, $b^2$ also always leaves a remainder of 1 when divided by 24.  Let $a^2 = 24m + 1$ and $b^2 = 24n + 1$ for some integers $m, n$.  Therefore, $a^2 - b^2 = (24m + 1) - (24n + 1) = 24(m-n)$, which is divisible by 24.

\item First, assume $n$ is a square number.  If $n = {p_1}^{a_1} \cdot  ... \cdot  {p_k}^{a_k}$, then $\sqrt n = {p_1}^{a_1/2} \cdot  ... \cdot  {p_k}^{a_k/2}$.  By the definition of prime factorization, $a_i$ and $a_i/2$ must be integers for all $i$ such that $1 \leq i \leq k$.  Thus, $a_i/2$ can only be an integer if 2 evenly divides $a_i$.  Therefore, each $a_i$ must be even.  Next, assume each $a_i$ is even.  Let $a_i = 2b_i$ for some integers $b_i$.  Thus, $n = {p_1}^{2b_1} \cdot  ... \cdot  {p_k}^{2b_k}$.  Since each $b_i$ is an integer, $\sqrt n = {p_1}^{b_1} \cdot  ... \cdot  {p_k}^{b_k}$ is an integer and $n$ must be a square number.

\item Let $a = {p_1}^{x_1} \cdot  ... \cdot  {p_k}^{x_k}$ and $b = {q_1}^{y_1} \cdot  ... \cdot  {q_j}^{y_j}$ be the prime factorizations of $a$ and $b$ respectively such that $p_m \neq q_n$ for all $m, n$.  As such, $ab = {p_1}^{x_1} \cdot  ... \cdot  {p_k}^{x_k} \cdot  {q_1}^{y_1} \cdot  ... \cdot  {q_j}^{y_j}$.  If $ab$ is a square, then $x_1, ..., x_k$ and $y_1, ..., y_j$ must be even.  Because $x_1, ..., x_k$ are even, $a$ must a square.  Similarly, since $y_1, ..., y_j$ are even, $b$ must also be a square.

\end{enumerate}
\end{document}
