\documentclass{article}
\usepackage{amsmath,amssymb}
\newcommand{\Z}{\mathbb{Z}}
\title{Homework 8}
\author{Madilyn Simons}
\date{}
\setlength{\parindent}{0pt}
\pdfpagewidth 8.5in
\pdfpageheight 11in

\usepackage{amssymb}
\usepackage{amsmath}

\begin{document}
\maketitle

\begin{enumerate}

\item
\begin{enumerate}

\item Since $(x^7 + 25x^6 - 25x + 5)$ is a nonconstant polynomial in
$\mathbb{Q}$, $\mathbb{Q}/(x^7 + 25x^6 - 25x + 5)$ is a field if and only if
$(x^7 + 25x^6 - 25x + 5)$ is irreducible in $\mathbb{Q}$.  We know that
$(x^7 + 25x^6 - 25x + 5)$ is irreducible in $\mathbb{Q}$ by Eisenstein's
Criterion for prime $5$.  Thus, $\mathbb{Q}/(x^7 + 25x^6 - 25x + 5)$ is a field.

\item Consider $\mathbb{Z}/2\mathbb{Z}$.  If $f(x) = x^3 + 2x^2 - x + 1$, then
$\overline{f}(x) = x^3 - x + 1$.  We know that $\overline{f}(x)$ is
irreducible in $\mathbb{Z}/2\mathbb{Z}$ because
$\overline{f}(0) = \overline{f}(1) = 1$.  Since $\overline{f}(x)$ is
irreducible in $\mathbb{Z}/2\mathbb{Z}$, $f(x)$ is irreducible in $\mathbb{Q}$,
which means that $\mathbb{Q}[x]/(x^3 + 2x^2 - x + 1)$ is a field.

\item Only first-degree and second-degree polynomials can be irreducible in
$\mathbb{R}[x]$.  Since $(x^5 + 42x^4 + \pi x^3 -1729x^2 + ln(2)x - 2019)$
is a fifth-degree polynomial, it is reducible in $\mathbb{R}[x]$.  Therefore
$\mathbb{R}[x]/(x^5 + 42x^4 + \pi x^3 -1729x^2 + ln(2)x - 2019)$ is NOT
a field.

\end{enumerate}

\item
\begin{enumerate}
\item
Let $f(x) = x^3 + 2x + 1$.  We know that $f(x)$ is irreducible in
$\mathbb{Z}/3\mathbb{Z}$ because $f(0) = f(1) = f(2) = 1$.  Therefore,
$K$ is a field.

\end{enumerate}

\item To prove that $(a,b) = (d)$, first let us prove that $(a,b) \subseteq (d)$.
Let $ar_1 + br_2$ be any element of $(a,b)$.  Since $d$ is the greatest common
divisor of $a$ and $b$, $a = dx$ and $b = dy$ for some integers $x$ and $y$.
Therefore, $ar_1 + br_2 = dxr_1 + dyr_2 = d(xr_{1} + dyr_{2})$, which is an
element of $(d)$.  Thus, $(a,b) \subseteq (d)$. \\ \\
Next, let us prove $(d) \subseteq (a,b)$.  Since $d$ is the greatest common
divisor of $a$ and $b$, $d = au + bv$ for some integers $u$ and $v$.  Let
$rd$ be any element of $(d)$.
Therefore, $rd = r(au + bv) = rau + rbv = a(ru) + b(rv) \subseteq (a,b)$.
Thus $(d) \subseteq (a,b)$. \\ \\
Since $(d) \subseteq (a,b)$ and $(a,b) \subseteq (d)$, $(a,b) = (d)$.

\item
Let $f(x) \in I$ and $g(x) \in \mathbb{Z}[x]$ such that
\[
f(x) = a_{0} + a_{1}x + a_{2}x^{2} + ... + a_{n-1}x^{n-1} + 2a_{n}x^n
\]
and
\[
g(x) = b_{0} + b_{1}x + b_{2}x^{2} + ... + b_{m-1}x^{m-1} + 2b_{m}x^m.
\]
Then,
\[
f(x)*g(x) = a_{0}b_{0} + (a_{0}b_{1} + a_{1}b_{0})x + (a_{0}b_{2}+a_{1}b{1}+a_{2}b_{0}) + ... + 2a_{n}b_{m}x^{n+m}
\]
and
\[
g(x)*f(x) = b_{0}a_{0} + (b_{0}a_{1} + b_{1}a_{0})x + (b_{0}a_{2}+b_{1}a{1}+b_{2}a_{0}) + ... + 2b_{m}a_{n}x^{m+n}.
\]
Since the leading coefficients of $f(x)*g(x)$ and $g(x)*f(x)$ are both even,
$f(x)*g(x) \in I$ and $g(x)*f(x) \in I$.  Therefore, $I$ is an ideal of $\mathbb{Z}[x]$.

\item Let $f(x)$, $g(x) \in I$ such that $g(x) \neq 0_{F}$.
By the Division Algorithm, $g(x) = f(x)q(x) + r(x)$ such that $r(x) = 0_{F}$
or $\deg r(x) < \deg g(x)$.  By the definition of $I$,
\[
g(7) = f(7)q(7) + r(7)
\]
\[
0 = f0*q(7) + r(7)
\]
\[
0 = r(7).
\]
Therefore, $r(x) \in I$ and $f(x)$ divides $g(x)$.  Therefore, $I$ is
finitely-generated.

\end{enumerate}
\end{document}
