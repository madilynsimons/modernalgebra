\documentclass{article}
\usepackage{amsmath,amssymb}
\newcommand{\Z}{\mathbb{Z}}
\title{Homework 8}
\author{Madilyn Simons}
\date{}
\setlength{\parindent}{0pt}
\pdfpagewidth 8.5in
\pdfpageheight 11in

\usepackage{amssymb}
\usepackage{amsmath}

\begin{document}
\maketitle

\begin{enumerate}

\item
\begin{enumerate}

\item Since $(x^7 + 25x^6 - 25x + 5)$ is a nonconstant polynomial in
$\mathbb{Q}$, $\mathbb{Q}/(x^7 + 25x^6 - 25x + 5)$ is a field if and only if
$(x^7 + 25x^6 - 25x + 5)$ is irreducible in $\mathbb{Q}$.  We know that
$(x^7 + 25x^6 - 25x + 5)$ is irreducible in $\mathbb{Q}$ by Eisenstein's
Criterion for prime $5$.  Thus, $\mathbb{Q}/(x^7 + 25x^6 - 25x + 5)$ is a field.

\item Consider $\mathbb{Z}/2\mathbb{Z}$.  If $f(x) = x^3 + 2x^2 - x + 1$, then
$\overline{f}(x) = x^3 - x + 1$.  We know that $\overline{f}(x)$ is
irreducible in $\mathbb{Z}/2\mathbb{Z}$ because
$\overline{f}(0) = \overline{f}(1) = 1$.  Since $\overline{f}(x)$ is
irreducible in $\mathbb{Z}/2\mathbb{Z}$, $f(x)$ is irreducible in $\mathbb{Q}$,
which means that $\mathbb{Q}[x]/(x^3 + 2x^2 - x + 1)$ is a field.

\item Only first-degree and second-degree polynomials can be irreducible in
$\mathbb{R}[x]$.  Since $(x^5 + 42x^4 + \pi x^3 -1729x^2 + ln(2)x - 2019)$
is a fifth-degree polynomial, it is reducible in $\mathbb{R}[x]$.  Therefore
$\mathbb{R}[x]/(x^5 + 42x^4 + \pi x^3 -1729x^2 + ln(2)x - 2019)$ is NOT
a field.

\end{enumerate}

\item
\begin{enumerate}
\item
Let $f(x) = x^3 + 2x + 1$.  We know that $f(x)$ is irreducible in
$\mathbb{Z}/3\mathbb{Z}$ because $f(0) = f(1) = f(2) = 1$.  Therefore,
$K$ is a field.

\item TODO

\end{enumerate}

\item To prove that $(a,b) = (d)$, first let us prove that $(a,b) \subseteq (d)$.
Let $ar_1 + br_2$ be any element of $(a,b)$.  Since $d$ is the greatest common
divisor of $a$ and $b$, $a = dx$ and $b = dy$ for some integers $x$ and $y$.
Therefore, $ar_1 + br_2 = dxr_1 + dyr_2 = d(xr_{1} + dyr_{2})$, which is an
element of $(d)$.  Thus, $(a,b) \subseteq (d)$. \\ \\
Next, let us prove $(d) \subseteq (a,b)$.  Since $d$ is the greatest common
divisor of $a$ and $b$, $d = au + bv$ for some integers $u$ and $v$.  Let
$rd$ be any element of $(d)$.
Therefore, $rd = r(au + bv) = rau + rbv = a(ru) + b(rv) \subseteq (a,b)$.
Thus $(d) \subseteq (a,b)$. \\ \\
Since $(d) \subseteq (a,b)$ and $(a,b) \subseteq (d)$, $(a,b) = (d)$.

\item
Let $a$, $b \in I$ such that
\[a = 2a_0 + 2a_{1}x + ... + 2a_{n}x^n
\]
and
\[b = 2b_0 + 2_b{1}x + ... 2b_{n}x^n
\]
and $a_i$ and $b_i$ are integers for all
$i$.  We know that $I$ holds under subtraction because
\begin{align*}
a - b &= (2a_0 + 2a_{1}x + ... + 2a_{n}x^{n}) - (2b_0 + 2_b{1}x + ... 2b_{n}x^n) \\
&= (2a_0 - 2b_0) + (2a_{1} - 2b_{1})x + ... + (2a_{n} - 2b_{n})x^n \\
&= 2(a_0 - b_0) + 2(a_{1} - b_{1})x + ... + 2(a_{n} - b_{n})x^n \\
\end{align*}

Next let $c = c_0 + c_{1}x + ... + c_{n}x^n$ be an element of $\mathbb{Z}[x]$.
We know that $I$ absorbs multiplication because
\begin{align*}
ca &= (c_0 + c_{1}x + ... + c_{n}x^{n})(2a_0 + 2a_{1}x + ...+ 2a_{n}x^{n}) \\
&= 2(c_0 + c_{1}x + ... + c_{n}x^{n})(a_0 + a_{1}x + ... + a_{n}x^{n}) \\
&= 2(c_{0}a_{0} + (c_{0}a_{1} + c_{1}a_{0})x + (c_{0}a_{2} + c_{1}a_{1} + c_{2}a_{0})x^{2}
+ ... + c_{n}a_{n}x^{n}) \\
&= 2c_{0}a_{0} + 2(c_{0}a_{1} + c_{1}a_{0})x + 2(c_{0}a_{2} + c_{1}a_{1} + c_{2}a_{0})x^{2}
+ ... + 2c_{n}a_{n}x^{n} \\
&\subseteq I
\end{align*}
and
\begin{align*}
ac &= (2a_0 + 2a_{1}x + ...+ 2a_{n}x^{n})(c_0 + c_{1}x + ... + c_{n}x^{n}) \\
&= 2(a_0 + a_{1}x + ... + a_{n}x^{n})(c_0 + c_{1}x + ... + c_{n}x^{n}) \\
&= 2(a_{0}c_{0} + (a_{0}c_{1} + a_{1}c_{0})x + (a_{0}c_{2} + a_{1}c_{1} + a_{2}c_{0})x^{2}
+ ... + a_{n}c_{n}x^{n}) \\
&= 2a_{0}c_{0} + 2(a_{0}c_{1} + a_{1}c_{0})x + 2(a_{0}c_{2} + a_{1}c_{1} + a_{2}c_{0})x^{2}
+ ... + 2a_{n}c_{n}x^{n} \\
&\subseteq I
\end{align*}

Therefore, $I$ is an ideal of $\mathbb{Z}[x]$.

\end{enumerate}
\end{document}
