\documentclass{article}
\usepackage{amsmath,amssymb}
\newcommand{\Z}{\mathbb{Z}}
\title{Homework 8}
\author{Madilyn Simons}
\date{}
\setlength{\parindent}{0pt}
\pdfpagewidth 8.5in
\pdfpageheight 11in

\usepackage{amssymb}
\usepackage{amsmath}

\begin{document}
\maketitle

\begin{enumerate}

\item
\begin{enumerate}

\item Let $I$ be a nonempty ideal in $F$ such that $I$ does not equal $F$.
Let $x \in F$ and $x \notin I$, let $a \in I$, and let $a^{-1}$ be the
multiplicative inverse of $a$.  If $a \in I$, then $ax \in I$ and
$a^{-1}ax \in I$.  However, $a^{-1}ax = 1_{F}x = x$ and $x \notin I$ so this
is a contradiction.

\item Since $f$ is a homomorphism of rings, the kernel of $f$ is an ideal
of $F$.  The only ideals of $F$ are $F$ and $(0)$.  If the kernel of
$f$ is $(0)$, then $f$ is injective.  If the kernel of $f$ is $F$, then
$f$ is the zero function.

\end{enumerate}

\item TODO Rewrite this and replace $(q_{1}(x))$ with $(q_{i}(x))$

First, assume $(p(x))$ is maximal.  Let
$p(x) = q_{1}(x)q_{2}(x) ... q_{n}(x)$ be the prime factorization of $p(x)$.
Therefore $(p(x)) \subset (q_{1}(x))$ since $q_{1}(x)$ divides $p(x)$.
Since $(p(x))$ is maximal, either $(p(x)) = (q_{1}(x))$ or $F[x] = (q_{1}(x))$.
\\ \\
Assume $(p(x)) = (q_{1}(x))$.  Therefore $p(x)$ must be some divisor of
$q_{1}(x)$ such that $q_{1}(x) = p(x)q_{2}^{-1}(x) ... q_{n}^{-1}(x)$.  This
implies that $q_{2}(x), q_{3}(x), ..., q_{n}(x)$ are units, which are nonzero
constants.  If $p(x)$ is the product of an irreducible polynomial $q_{1}(x)$
and several nonzero constant polynomials, then $p(x)$ is irreducible.
\\ \\
Now assume $F[x] = (q_{1}(x))$.  If this is the case, then $q_{1}(x)$ must
be a unit, which is a contradiction.
\\ \\
Therefore if $(p(x))$ is maximal, then $p(x)$ is irreducible.  Then the quotient
ring $F[x]/(p(x))$ is all
\\ \\
Next, assume $p(x)$ is irreducible.  TODO

\item
\begin{enumerate}
\item
By the Division Algorithm, $a(x) = b(x)q(x) + r(x)$ such that
$0 \leq deg r(x) < deg b(x)$.  By Theorem 6.1, $b(x)q(x) \in I$ and
$a(x) - b(x)q(x) = r(x) \in I$.

\item Let $a(x)$ by any nonzero polynomial in $I$.  By the Division Algorithm,
$a(x) = p(x)q(x) + r(x)$ such that $0 \leq deg r(x) < deg p(x)$.  However,
$p(x)$ is of minimal degree, so $deg r(x) = 0$.  Therefore, $p(x)$ divides
all $a(x) \in I$.

\end{enumerate}

\end{enumerate}
\end{document}
