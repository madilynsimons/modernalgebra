\documentclass{article}
\usepackage{amsmath,amssymb}
\newcommand{\Z}{\mathbb{Z}}
\title{Homework 9}
\author{Madilyn Simons}
\date{}
\setlength{\parindent}{0pt}
\pdfpagewidth 8.5in
\pdfpageheight 11in

\usepackage{amssymb}
\usepackage{amsmath}

\begin{document}
\maketitle

\begin{enumerate}

\item
\begin{enumerate}

\item Let $I$ be a nonempty ideal in $F$ such that $I$ does not equal $F$.
Let $x \in F$ and $x \notin I$, let $a \in I$, and let $a^{-1}$ be the
multiplicative inverse of $a$.  If $a \in I$, then $ax \in I$ and
$a^{-1}ax \in I$.  However, $a^{-1}ax = 1_{F}x = x$ and $x \notin I$ so this
is a contradiction.

\item Since $f$ is a homomorphism of rings, the kernel of $f$ is an ideal
of $F$.  The only ideals of $F$ are $F$ and $(0)$.  If the kernel of
$f$ is $(0)$, then $f$ is injective.  If the kernel of $f$ is $F$, then
$f$ is the zero function.

\end{enumerate}

\item TODO Rewrite this and replace $(q_{1}(x))$ with $(q_{i}(x))$

First, assume $(p(x))$ is maximal.  Let
$p(x) = q_{1}(x)q_{2}(x) ... q_{n}(x)$ be the prime factorization of $p(x)$.
Therefore $(p(x)) \subset (q_{1}(x))$ since $q_{1}(x)$ divides $p(x)$.
Since $(p(x))$ is maximal, either $(p(x)) = (q_{1}(x))$ or $F[x] = (q_{1}(x))$.
\\ \\
Assume $(p(x)) = (q_{1}(x))$.  Therefore $p(x)$ must be some divisor of
$q_{1}(x)$ such that $q_{1}(x) = p(x)q_{2}^{-1}(x) ... q_{n}^{-1}(x)$.  This
implies that $q_{2}(x), q_{3}(x), ..., q_{n}(x)$ are units, which are nonzero
constants.  If $p(x)$ is the product of an irreducible polynomial $q_{1}(x)$
and several nonzero constant polynomials, then $p(x)$ is irreducible.
\\ \\
Now assume $F[x] = (q_{1}(x))$.  If this is the case, then $q_{1}(x)$ must
be a unit, which is a contradiction.
\\ \\
Next, assume $p(x)$ is irreducible.  Since $p(x)$ is irreducible, the quotient
ring $F[x]/(p(x))$ is a field and therefore $(p(x))$ is a maximal ideal.

\item
\begin{enumerate}
\item
By the Division Algorithm, $a(x) = b(x)q(x) + r(x)$ such that
$0 \leq deg r(x) < deg b(x)$.  By Theorem 6.1, $b(x)q(x) \in I$ and
$a(x) - b(x)q(x) = r(x) \in I$.

\item Let $a(x)$ by any nonzero polynomial in $I$.  By the Division Algorithm,
$a(x) = p(x)q(x) + r(x)$ such that $0 \leq deg r(x) < deg p(x)$.  However,
$p(x)$ is of minimal degree, so $deg r(x) = 0$.  Therefore, $p(x)$ divides
all $a(x) \in I$.

\end{enumerate}

\item
\begin{enumerate}
\item
Let $I = \{f(x) \in F[x] | f(\alpha) = 0\}$.  We know that $I$ is an ideal of
$F[x]$ because for all $f_{1}(x)$, $f_{2}(x) \in I$,
\[
f_{1}(\alpha) - f_{2}(\alpha) = 0-0 = 0
\]
So $f_{1}(x) - f_{2}(x) \in I$.  For all $g(x) \in F[x]$ and $f(x) \in I$,
\[
f(\alpha)g(\alpha) = 0*g(\alpha) = 0
\] \[
g(\alpha)f(\alpha) = g(\alpha)*0 = 0
\]
So $f(x)g(x) \in I$ and $g(x)f(x) \in I$, and $I$ is a proper ideal of $F[x]$.
\\ \\
Since $\alpha$ is a root of $p(x)$, $p(\alpha) = 0$ and $p(\alpha) \in I$.
Because all ideals of $F[x]$ are principal ideals and $p(x)$ is irreducible,
$I = (p(x))$ and all elements of $I$ are divisible by $p(x)$.  If $g(\alpha) = 0$,
then $g(x) \in I$ and $p(x) | g(x)$.
\\
\\
Next, assume $p(x) | g(x)$.  Then $g(x) = p(x)q(x)$ for some $q(x)$, and so
\[
g(\alpha) = p(\alpha)q(\alpha) = 0.
\]
Thus, $g(\alpha) = 0$ if and only if $p(x) | g(x)$.

\item Let $I = \{f(x) \in \mathbb{Q} | f(\sqrt[7]{2}) = 0\}$.
Let $p(x) = x^2 - 2$.  By Eisenstein's Criterion, $x^7 - 2$ is irreducible, and
$p(\sqrt[7]{2}) = 0$ so $p(x) \in I$.  By problem $4a$, any nonzero
$g(x) \in I$ must be divisible by $p(x)$, meaning that $deg$ $g \geq 7$.

\item Let $g:F[x] \rightarrow F[\alpha]$ be a homomorphism of rings
given by $g(f(x)) = f(\alpha)$ such that $f(x) \in F[x]$.  Then the kernel of
$g$ is $(p(x))$.  We know that $g$ is surjective, because all $f(\alpha)$ map to
some $f(x)$ so that there is an $f(x)$ for all $f(\alpha)$ such that
$g(f(x)) = f(\alpha)$.  Thus, by Theorem 6.13, $F[x]/(p(x))$ is isomorphic
to $F[\alpha]$.

\end{enumerate}
\end{enumerate}
\end{document}
