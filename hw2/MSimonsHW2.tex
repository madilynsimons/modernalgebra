\documentclass{article}
\title{Homework 2}
\author{Madilyn Simons}
\date{}
\setlength{\parindent}{0pt}
\pdfpagewidth 8.5in
\pdfpageheight 11in

\usepackage{amssymb}
\usepackage{amsmath}

\begin{document}
\maketitle

\begin{enumerate}

\item By definition, $\binom{p}{k} = \frac{p!}{k!(p-k)!} = p\frac{(p-1)!}{k!(p-k)!}$.
Since $k < p$ and $p-k < p$, neither $k!$ nor $(p-k)!$ have any prime factors that divide $p$,
and $\frac{p!}{k!(p-k)!}$ is an integer, $\frac{(p-1)!}{k!(p-k)!}$ must also be an integer.
This implies that $p \vert \binom{p}{k}$.

\item By definition of binomial coefficients,
\[
(a+b)^p = \sum_{k=0}^{p} \binom{p}{k} a^{p-k}b^k = (a^p + b^p) +
\sum_{k=1}^{p-1} (\binom{p}{k} a^{p-k}b^k).
\]
Since $p \vert \binom{p}{k}$ for all $k < p$ and all numbers between 1 and $p$-1
(inclusive) as less than $p$, $\sum_{k=0}^{p} \binom{p}{k} a^{p-k}b^k$ is divisible
by $p$.  This means that $\sum_{k=0}^{p} \binom{p}{k} a^{p-k}b^k \equiv 0$ (mod p).
Consequently,
\[
(a^p + b^p) + \sum_{k=1}^{p-1} (\binom{p}{k} a^{p-k}b^k) \equiv (a^p + b^p) + 0
\equiv a^p + b ^p (\textrm{mod}\ p).
\]

\end{enumerate}
\end{document}
