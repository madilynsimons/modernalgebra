\documentclass{article}
\title{Homework 2}
\author{Madilyn Simons}
\date{}
\setlength{\parindent}{0pt}
\pdfpagewidth 8.5in
\pdfpageheight 11in

\usepackage{amssymb}
\usepackage{amsmath}

\begin{document}
\maketitle

\begin{enumerate}

\item Since $k < p$ and $(p-k) < p$, neither $k!$ nor $(p-k)!$ have any prime factors that divide $p$.
Because of this and the fact that $\frac{p!}{k!(p-k)!}$ is an integer,
$\frac{(p-1)!}{k!(p-k)!}$ must also be an integer.
By definition, $\binom{p}{k} = \frac{p!}{k!(p-k)!} = p\frac{(p-1)!}{k!(p-k)!}$.
Because $p$ and $\frac{(p-1)!}{k!(p-k)!}$ are both integers, this implies that $p \vert \binom{p}{k}$.

\item By definition of binomial coefficients,
\[
(a+b)^p = \sum_{k=0}^{p} \binom{p}{k} a^{p-k}b^k = (a^p + b^p) +
\sum_{k=1}^{p-1} (\binom{p}{k} a^{p-k}b^k).
\]
Since $p \vert \binom{p}{k}$ for all $k < p$ and all numbers between 1 and $p-1$
(inclusive) are less than $p$, $\sum_{k=1}^{p-1} \binom{p}{k} a^{p-k}b^k$ is divisible
by $p$.  This means that $\sum_{k=1}^{p-1} \binom{p}{k} a^{p-k}b^k \equiv 0$ (mod $p$).
Consequently,
\[
(a^p + b^p) + \sum_{k=1}^{p-1} (\binom{p}{k} a^{p-k}b^k) \equiv (a^p + b^p) + 0
\equiv a^p + b ^p (\textrm{mod}\ p).
\]

\item Let $a$ be some element of $\mathbb{Z}/m\mathbb{Z}$.  Assume $a$ is a unit
and let $a^{-1}$ be its inverse.  Also assume $a$ is a zero divisor and let
$ab \equiv 0$ (mod $m$) for some nonzero element $b$ of $\mathbb{Z}/m\mathbb{Z}$.
As such,
\[
a^{-1}ab \equiv (a^{-1}a)b \equiv 1b \equiv b (\textrm{mod}\ m)
\]
and
\[
a^{-1}ab \equiv a^{-1}(ab) \equiv a^{-1}(0) \equiv 0 (\textrm{mod}\ m)
\]
Therefore, $b \equiv 0$ (mod $m$).  This is a contradiction.  Therefore, $a$ cannot
be a zero divisor and a unit.

\item Let $a$ be some nonzero element of $\mathbb{Z}/m\mathbb{Z}$.  Either $(a,m) = 1$
or $(a,m) > 1$.  First, let $(a,m) = 1$.  If $(a,m) = 1$, then $a$ is a unit
and we are done.  Next, let $(a, m) = c$ for some $c$ such that $c > 1$.
Let $a = {p_0}^{a_0}{p_1}^{a_1} ... {p_k}^{a_k}$ be the prime factorization of
$a$ such that $a_i \geq 0$ for all $i$.
Similarly, let $m = {p_0}^{m_0}{p_1}^{m_1} ... {p_k}^{m_k}$ be the prime factorization of
$m$ such that $m_i \geq 0$ for all $i$.
Since $a$ and $m$ are not relatively prime and $a$ is a nonzero element, there exists
some $d = {p_0}^{max(a_0, m_0)} ... {p_k}^{max(a_k, m_k)}$, which is divisible by $a$
and is a zero element of $\mathbb{Z}/m\mathbb{Z}$.
Let $x = {p_0}^{x_0} ... {p_k}^{x_k}$ such that ${x_i}{a_i} = max(a_i, m_i)$
for all $i$.  That is, $ax \equiv d \equiv 0$ (mod m).
Therefore, if $(a, m) \neq 1$, then $a$ is a zero divisor.

\item Let $ua = 1$ and $ub = 1$ for elements $a, b$ in $\mathbb{Z}/m\mathbb{Z}$.
As such,
\[
uab = (ua)b = 1b = b
\]
and
\[
uab = (ub)a = 1a = a.
\]
Therefore, $a = b$, proving that $u$ has exactly one inverse.

\end{enumerate}
\end{document}
