\documentclass{article}
\usepackage{amsmath,amssymb}
\newcommand{\Z}{\mathbb{Z}}
\title{Homework 7}
\author{Madilyn Simons}
\date{}
\setlength{\parindent}{0pt}
\pdfpagewidth 8.5in
\pdfpageheight 11in

\usepackage{amssymb}
\usepackage{amsmath}

\begin{document}
\maketitle

\begin{enumerate}

\item If $f(x) = x^7 - 7x + 20$, then
\[
f(x+1) = x^7 + 7x^6 + 21x^5 + 35x^4 + 35x^3 + 21x^2 + 14
\]
By Eisenstein's Criterion, for $p = 7$, $f(x + 1)$ is irreducible in $\mathbb{Q}[x]$,
and so $f(x)$ is irreducible in $\mathbb{Q}[x]$.

\item Let $f(x) = x^5 - 4x^4 + 4x^3 + 3x^2 -26x + 21$.
If $p = 2$, then $\overline{f}(x) = x^5 + x^2 + 1$ in $\mathbb{Z}_{p}[x]$.
We know that $\overline{f}(x)$ is irreducible in $\mathbb{Z}_{p}[x]$ because
since it has no roots, it is either irreducible or the product of an irreducible
quadratic and an irreducible cubic.  However, the only irreducible quadratic
in $\mathbb{Z}_{p}[x]$ is $x^2 + x + 1$, which does not divide $\overline{f}(x)$.
Therefore, $f(x)$ is irreducible in $\mathbb{Q}[x]$.

\item
\begin{enumerate}
\item
Since $-1 + 3i$ is a root of $f(x)$, $-1 - 3i$ is also a root of $f(x)$.
Therefore
\[
f(x) = (x + 1 + 3i)(x + 1 - 3i)(h(x)) = (x^2 + 2x + 10)(h(x))
\] for
some polynomial $h(x)$.
\\ \\
By division,
\[h(x) = f(x) / (x^2 + 2x + 10) = x^4 + x^2 - 6.
\]
Thus,
\[f(x) = (x^2 + 2x + 10)(x^4 + x^2 - 6) = (x^2 + 2x + 10)(x^2 + 3)(x^2 - 2)
\]
We know that $(x^2 + 2x + 10)$ and $(x^2 - 2)$ are irreducible in $\mathbb{Q}[x]$
by Eisenstein's Criterion for prime $p = 2$, and $(x^2 + 3)$ is irreducible
in $\mathbb{Q}[x]$ by Eisenstein's Criterion for prime $p = 3$.
Therefore, the irreducible
factorization of $f(x)$ in $\mathbb{Q}[x]$ is
\[
f(c) = (x^2 + 2x + 10)(x^2 + 3)(x^2 - 2).
\]

\item
In $\mathbb{R}[x]$,
\[f(x) = (x^2 + 2x + 10)(x^2 + 3)(x^2 - 2)
= (x^2 + 2x + 10)(x^2 + 3)(x + \sqrt{2})(x - \sqrt{2})
\]
We know that $(x^2 + 2x + 10)$ and $(x^2 + 3)$ are irreducible $\mathbb{R}[x]$
because a polynomial of the form $f(x) = ax^2 + bx + c$ is irreducible in
$\mathbb{R}[x]$ if $b^2 - 4ac < 0$.  Therefore, the irreducible factorization
of $f(x)$ in $\mathbb{R}[x]$ is
\[
(x^2 + 2x + 10)(x^2 + 3)(x + \sqrt{2})(x - \sqrt{2}).
\]

\item
In $\mathbb{C}[x]$, the irreducible factorization of $f(x)$ is
\[
f(x) = (x + 1 + 3i)(x + 1 - 3i)(x + \sqrt{3}i)(x - \sqrt{3}i)(x + \sqrt{2})(x - \sqrt{2}).
\]
We know that each of these polynomials are irreducible in $\mathbb{C}[x]$ because
they all have degree 1.

\end{enumerate}
\end{enumerate}
\end{document}
