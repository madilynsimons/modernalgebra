\documentclass{article}
\title{Homework 5}
\author{Madilyn Simons}
\date{}
\setlength{\parindent}{0pt}
\pdfpagewidth 8.5in
\pdfpageheight 11in

\usepackage{amssymb}
\usepackage{amsmath}

\begin{document}
\maketitle

\begin{enumerate}

\item Assume that if $f(x)$, $g(x)$, $s(x)$, $d(x)$, $p(x)$ in $S[x]$ satisfy
\[
f(x) + g(x) = s(x)
\] \[
f(x) - g(x) = d(x)
\] \[
f(x)g(x) = p(x)
\]
then, for any $\alpha$ in $R$,
\[
\] \[
f(\alpha) + g(\alpha) = s(\alpha)
\] \[
f(\alpha) - g(\alpha) = d(\alpha)
\] \[
f(\alpha)g(\alpha) = p(\alpha)
\]
We can prove that $\phi_{\alpha}$ preserves addition as such:
\[
\phi_{\alpha}(f) + \phi_{\alpha}(g) = f(\alpha) + g(\alpha) = s(\alpha) = \phi_{\alpha}(s)
\] \[
\phi_{\alpha}(f) - \phi_{\alpha}(g) = f(\alpha) - g(\alpha) = d(\alpha) = \phi_{\alpha}(d)
\]
We can prove that $\phi_{\alpha}$ preserves multiplication as such:
\[
\phi_{\alpha}(f)\phi_{\alpha}(g) = f(\alpha)g(\alpha) = p(\alpha) = \phi_{\alpha}(p)
\]
Thus, $\phi_{\alpha}$ is a ring homomorphism.

\item Let $f(\alpha)$, $g(\alpha)$ be elements in $S[\alpha]$.  This implies
that $f(x)$ and $g(x)$ are elements of $S[x]$.  Since $S$ is a ring, its
elements are closed under addition, implying that $S[x]$ preserves
addition as well.  Let $f(x) + g(x) = s(x)$ and $f(x) - g(x) = d(x)$.  Since
$s(x)$ and $d(x)$ are in $S[x]$, $s(\alpha) = f(\alpha) + g(\alpha)$ and
$d(\alpha) = f(\alpha) - g(\alpha)$ are elements of $S[\alpha]$ and $S[\alpha]$
preserves addition.
\\
\\
Similarly, since $S$ preserves multiplication, we can assume $p(x) = f(x)g(x)$
is an element of $S[x]$.  Therefore $p(\alpha) = f(\alpha)g(\alpha)$ is an
element of $S[\alpha]$ and $S[alpha]$ preserves multiplication.
\\
\\
Thus, $S[\alpha]$ is a subring of $R$.




\item Let $a(x) = (a_0 + a_{1}x + ... + a_{n}x^n)$, $b(x) = (b_0 + b_{1}x + ... + b_{n}x^m)$
be elements of $R_{1}[x]$.
\\ \\
Since $F$ is a ring homomorphism, we can prove $G$ preserves addition as such:

\begin{align*}
G(a(x) + b(x)) &= G((a_0 + a_{1}x + ... + a_{n}x^{n}) + (b_0 + b_{1}x + ... + b_{m}x^{m})) \\
&= G(a_0 + a_{1}x + ... + a_{n}x^n + b_0 + b_{1}x + ... + b_{m}x^{m}) \\
&= F(a_{0}) + F(a_{1})x + ... + F(a_{n})x^{n} + F(b_{0}) + F(b_{1})x + ... + F(b_{m})x^{m} \\
&= (F(a_{0}) + F(a_{1})x + ... + F(a_{n})x^{n}) + (F(b_{0}) + F(b_{1})x + ... + F(b_{m})x^{m}) \\
&= G(a_0 + a_{1}x + ... + a_{n}x^{n}) + G(b_0 + b_{1}x + ... + b_{m}x^{m}) \\
&= G(a(x)) + G(b(x))
\end{align*}

We can prove $G$ preserves multiplication as such:
\begin{align*}
G((ab)(x)) &= G((a_0 + a_{1}x + ... + a_{n}x^{n})(b_0 + b_{1}x + ... + b_{m}x^{m})) \\
&= G(a_{0}b_{0} + (a_{0}b_{1} + a_{1}b_{0})x + (a_{0}b_{2} + a_{1}b_{1} + a_{2}b_{0})x^2 + ... + a_{n}b_{m}x^{x+m}) \\
&= F(a_{0}b_{0}) + F(a_{0}b_{1} + a_{1}b_{0})x + F(a_{0}b_{2} + a_{1}b_{1} + a_{2}b_{0})x^2 + ... + F(a_{n}b_{m})x^{x+m} \\
&= (F(a_{0}) + F(a_{1})x + ... + F(a_{n})x^{n})(F(b_{0}) + F(b_{1})x + ... + F(b_{m})x^{m}) \\
&= G(a_0 + a_{1}x + ... + a_{n}x^{n})G(b_0 + b_{1}x + ... + b_{m}x^{m}) \\
&= G(a(x))G(b(x))
\end{align*}

Therefore $G$ is also a a ring homomorphism.

\end{enumerate}
\end{document}
