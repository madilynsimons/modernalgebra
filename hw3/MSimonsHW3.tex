\documentclass{article}
\title{Homework 3}
\author{Madilyn Simons}
\date{}
\setlength{\parindent}{0pt}
\pdfpagewidth 8.5in
\pdfpageheight 11in

\usepackage{amssymb}
\usepackage{amsmath}

\begin{document}
\maketitle

\begin{enumerate}

\item The set $\frac{1}{2}\mathbb{Z}$ is not a ring.  A ring must be closed under
multiplication.  Let $a = \frac{m}{2}$ and $b = \frac{n}{2}$ for some integers
$m$, $n$.  We know that $ab = \frac{mn}/4$.  Assume $m$ and $n$ are both odd and
that $m = 2x + 1$ and $n = 2y + 1$ for some integers $x$, $y$.  Therefore,
\[
ab = \frac{mn}{4} = \frac{(2x+1)(2y+1)}{4} = \frac{2(2xy+y)+1}{4}
\]
Since $ab$ is not an element of the set, this means that the set is not closed
under multiplication.

\item This set is a ring. \\
Let $a = \frac{m}{2x+1}$, $b=\frac{n}{2y+1}$, $c=\frac{l}{2z+1}$ be elements of
the set for some integers $m$, $n$, $l$, $x$, $y$, and $z$.\\
The set is closed under addition:
\[
a + b = \frac{m}{2x+1} + frac{n}{2y+1} = \frac{m(2y+1) + 2(nx+1)}{2(2xy+x+y)+1}
\]
Associative addition holds:
\[
a+(b+c) = \frac{m}{2x+1} + (\frac{n}{2y+1} + \frac{l}{2z+1})
\]
\[
= \frac{m(2y+1)(2z+1) + n(2z+1)(2x+1) + l(2y+1)(2x+1)}{(2x+1)(2y+1)(2z+1)}
\]
\[
= (\frac{m}{2x+1} + \frac{n}{2y+1}) + \frac{l}{2z+1} = (a+b)+c
\]
Commutative addition holds:
\[
a + b = \frac{m}{2x+1} + \frac{n}{2y+1}
\]
\[
= \frac{m(2y+1) + 2(nx+1)}{2(2xy+x+y)+1}
\]
\[
= \frac{n}{2y+1} + \frac{m}{2x+1} = b + a
\]
There exists an $0$ element in the set such that:
\[
a + 0 = \frac{m}{2x+1} + \frac{0}{1} = \frac{m}{2x+1} = a
= \frac{0}{1} + \frac{m}{2x+1} = 0 + a
\]
There is a solution to $a + x = 0$.  Let $x = \frac{-m}{2x+1}$:
\[
a + x = \frac{m}{2x+1} + \frac{-m}{2x+1} = \frac{m + {-m}}{2x+1} = \frac{0}{2x+1} = 0
\]
The set is closed under multiplication:
\[
ab = \frac{m}{2x+1}\cdot\frac{n}{2y+1} = \frac{mn}{(2x+1)(2y+1)} = \frac{mn}{2(2xy+x+y)+1}
\]
Associative multiplication holds:
\[
a(bc) = \frac{m}{2x+1}\cdot(\frac{n}{2y+1}\cdot\frac{l}{2z+1}) = (\frac{m}{2x+1}\cdot\frac{n}{2y+1})\cdot\frac{l}{2z+1}
= (ab)c
\]
The Distributive Property holds:
\[
a(b+c) = \frac{m}{2x+1}\cdot(\frac{n}{2y+1}+\frac{l}{2z+1})
= \frac{m}{2x+1}\cdot(\frac{n(2z+1) + l(2y+1)}{(2y+1)(2z+1)})
\]
\[
= \frac{m}{2x+1}\cdot\frac{n}{2y+1} + \frac{m}{2x+1}\cdot\frac{l}{2z+1}
=ab+ac
\]

\item The set is not a ring as it is not closed under multiplication.\\
Let $\frac{m}{6x+3}$, $\frac{n}{6y+4}$ be elements of the set for some integers
$m$, $n$, $x$, and $y$.  As such,
\[
\frac{m}{6x+3}\cdot\frac{n}{6y+4} = \frac{mn}{6(6xy+4x+3y+2)}
\]

\item 

\item Assume $M$ is a unit.  Therefore there exists some $M^{-1} \in M_2(\mathbb{Z})$
such that $MM^{-1} = I_2$.  We find $M^{-1}$ using Gaussian Elimination:
\[
\left[
\begin{array}{cc|cc}
a & b & 1 & 0 \\
c & d & 0 & 1
\end{array}
\right]
\Leftrightarrow
\left[
\begin{array}{cc|cc}
a & b & 1 & 0 \\
0 & \frac{ad-bc}{a} & \frac{-c}{a} & 1
\end{array}
\right]
\]
\[
\Leftrightarrow
\left[
\begin{array}{cc|cc}
a & b & 1 & 0 \\
0 & 1 & \frac{-c}{ad-bc} & \frac{a}{ad-bc}
\end{array}
\right]
\Leftrightarrow
\left[
\begin{array}{cc|cc}
a & 0 & \frac{ad}{ad-bc} & \frac{-ab}{ad-bc} \\
0 & 1 & \frac{-c}{ad-bc} & \frac{a}{ad-bc}
\end{array}
\right]
\]
\[
\Leftrightarrow
\left[
\begin{array}{cc|cc}
1 & 0 & \frac{d}{ad-bc} & \frac{-b}{ad-bc} \\
0 & 1 & \frac{-c}{ad-bc} & \frac{a}{ad-bc}
\end{array}
\right]
\]

Therefore $M^{-1} = \begin{bmatrix} \frac{d}{ad-bc} & \frac{-b}{ad-bc} \\
\frac{-c}{ad-bc} & \frac{a}{ad-bc}\end{bmatrix}$.  Since $M^{-1} \in M_2(\mathbb{Z})$,
each of its elements must be integers.  Its elements can only be
integers if $ad-bc$ evenly divides $a$, $b$, $c$, and $d$. \\
\\
Let $m$ be the greatest common denominator of $a$, $b$, $c$, and $d$.
Let $a = mx$, $b = my$, $c = mz$, and $d = mw$ for some integers
$x$, $y$, $z$, and $w$ that are not divisible by $m$.
As such $ad-bc = mxmw - mymz = m^2(xw - yz)$.  Since $m$ is the greatest common
denominator and $m^2$ is also a common denominator, $m^2 \leq m$ and this is
only possible if $m = 1$.  Therefore, the greatest common denominator of
$a$, $b$, $c$, and $d$ is 1.  If this is the case, $ad-bc$ must be $\pm1$ since
only $\pm1$ can evenly $a$, $b$, $c$, and $d$.
\\
\\
Next, assume $ad - bc = \pm1$.  By the last proof, $M$ can only be a unit if
$MM^{-1} = I_2$ and $M^{-1} \in M_2(\mathbb{Z})$.  If $ad - bc = 1$, then
$M^{-1} = \begin{bmatrix} d & -b \\ -c & a \end{bmatrix}$ and $MM^{-1} = I_2$.
If $ad - bc = -1$, then $M^{-1} = \begin{bmatrix} -d & b \\ c & -a \end{bmatrix}$
and $MM^{-1} = I_2$.
\\
\\
Thus, $M \in M_2(\mathbb{Z})$ is a unit if and only if $ad - bc = \pm1$.


\end{enumerate}
\end{document}
