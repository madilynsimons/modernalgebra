\documentclass{article}
\usepackage{amsmath,amssymb}
\newcommand{\Z}{\mathbb{Z}}
\title{Homework 6}
\author{Madilyn Simons}
\date{}
\setlength{\parindent}{0pt}
\pdfpagewidth 8.5in
\pdfpageheight 11in

\usepackage{amssymb}
\usepackage{amsmath}

\begin{document}
\maketitle

\begin{enumerate}

\item To prove that "is an associate of" is an equivalence relation on $R$, we
must show that it is reflexive, symmetric, and transitive.
\\
\\
First, since $R$ is a commutative ring, there exists some identity element in
$R$, $1_R$, such that $a1_R = a$.  We can say that $1_R$ is a unit since
$(1_{R})  (1_{R}) = 1_{R}$.  Since $a1_R = a$, $a$ is an associate of $a$
and "is an associate of" is reflexive.
\\
\\
Next, let $a$, $b \in R$ such that $a$ is an associate of $b$.  Therefore,
there exist some unit, $u$, in $R$ such that $au = b$.  Since $u$ is a unit,
there also exists some unit $v \in R$ such that $uv = 1_R$.
\\
\\
If $a = bu$, then
\[
av = (bu)v
\] \[
av = buv
\] \[
av = b(uv)
\] \[
av = b1_{R}
\] \[
av = b
\]
Therefore, $b$ is an associate of $a$ and "is an associate of" is reflexive.
\\
\\
Finally, let $a$ be an associate of $b$ and let $b$ be an associate of $c$ for
some elements $a$, $b$, $c \in R$.  Therefore, $a = bu$ and $b = cv$ for some
units $u$, $v \in R$.
\\
\\
We notice that
\[
a = bu
\] \[
a = (cv)u
\] \[
a = c(vu)
\]
We know that $vu$ is a unit because there exists some
$v^{-1}$, $u^{-1} \in R$ such that $vv^{-1} = 1_R$ and $uu^{-1} = 1_R$ since
$v$ and $u$ are units.  Thus, $(vu)(v^{-1}u^{-1}) = (vv^{-1})(uu^{-1}) = 1_R$
and so $vu$ is a unit.  Since $vu$ is a unit, $a$ is an associate of $c$ and
"is an associate of" is transitive.

\item
\begin{enumerate}
\item There are 4 different quadratic polynomial in $(\Z/2\Z)[x]$:
\[
x^2
\] \[
x^2 + 1
\] \[
x^2 + x
\] \[
x^2 + x + 1
\]
Let $f(x)$ be any quadratic polynomial in $(\Z/2\Z)[x]$.  We know that
if $f(x) = x^2$, then $f(x)$ is reducible because $0$ is a root.
If $f(x) = x^2 + 1$, then $f(x)$ is reducible because $1$ is a root.
If $f(x) = x^2 + x$, then $f(x)$ is reducible because $1$ is a root.
\\ \\
If $f(x) = x^2 + x + 1$, then $f(x)$ is irreducible because $f(x)$ has no roots
in $(\Z/2\Z)[x]$:
\[
f(0) = 0^2 + 0 + 1 = 1
\] \[
f(1) = 1^1 + 1 + 1 = 3 = 1
\]
\\
Therefore the only irreducible quadratic polynomial in $(\Z/2\Z)[x]$ is
$x^2 + x + 1$.

\item No, because $f(1) = 0$.
\item Yes, because $g(0) = g(1) = 1$.
\item Yes, because $h(0) = h(1) = 1$.

\end{enumerate}

\item By Eisenstein's Criterion, $p(x)$ is irreducible.  Let $q = 2$.  Since
$q$ is a prime that divides all of the coefficients except for the leading
coefficient and $q^2$ does not divide 34, $p(x)$ is irreducible.

\item Let $p = 11$.  Then $\overline{q}(x) = x^4 + 7x + 5$.
Since $\overline{q}(x)$ is irreducible in $(\Z/p\Z)[x]$ and
$p$ does not divide the leading coefficient of $q(x)$,
$q(x)$ is irreducible.

\end{enumerate}
\end{document}
