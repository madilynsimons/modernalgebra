\documentclass{article}
\title{Homework 4}
\author{Madilyn Simons}
\date{}
\setlength{\parindent}{0pt}
\pdfpagewidth 8.5in
\pdfpageheight 11in

\usepackage{amssymb}
\usepackage{amsmath}

\begin{document}
\maketitle

\begin{enumerate}

\item Consider ${\mathbb{Z}}_4 = {0, 1, 2, 3}$.  We know 0 is not a zero divisor since
a zero divisor is a nonzero element by definition.  We know that 1 is not a zero
because $1x = x$ for all $x$  and then $x$ must be 0 if $1x = 0$.  We know that
2 IS a zero divisor because $2\cdot2 = 4 = 0$.  We know that 3 is not a zero
divisor because 3 and 4 are relatively prime, meaning any zero element of
${\mathbb{Z}}_4$ must be a multiple of 4, which equals 0.  Therefore, there
does exist a ring with one zero divisor.

\item First, let us prove that $\cong$ is symmetric.
Since $f: R \rightarrow S$ is an isomorphism, this means that $f$ is bijective.
Let $f(r) = s$ for some $r \in R$ and some $s \in S$.
If $f$ is bijective, there exists some function $g: S \rightarrow R$ such that
$g(s) = g(f(r)) = r$. \\ \\
To prove $g$ is injective, let $s, s' \in S$ and $r, r' \in R$ such that
$s \neq s'$, $r \neq r'$, $f(s) = r$, and $f(s') = r'$.  As such,
$g(s) = g(f(r)) = r$ and $g(s') = g(f(r')) = r'$.  Since $r \neq r'$, this
implies that if $s \neq s'$, then $r \neq r'$ and $g$ must be injective. \\ \\
Because Im $f$ = $S$, there is an $r \in R$ for all $s \in S$ such that
$s = f(r)$.  Because $g(s) = g(f(r)) = r$, Im $g$ = $R$ and $g$ is surjective. \\ \\
We can prove that $g(s) + g(s') = g(s + s')$ as such: \\
\[
g(s) + g(s') = g(f(r)) + g(f(r')) = r + r' = g(f(r + r')) = g(f(r) + f(r')) = g(s + s')
\]
\\
We can prove $g(s)g(s') = g(ss')$ as such:
\[
g(s)g(s') = g(f(r))g(f(r')) = rr' = g(f(rr')) = g(f(r)f(r')) = g(ss')
\]
\\
Therefore, $g$ is an isomorphism and $\cong$ is symmetric.
\\ \\
Next, we prove that $\cong$ is transitive.  If $f$ and $g$ are isomorphisms,
then there exists $r \in R$, $s \in S$, and $t \in T$ such that
$f(r) = s$ and $g(s) = t$.  As such, let $h: R \rightarrow T$ be the function
such that $h(r) = g(f(r)) = g(s) = t$. \\ \\
Let $g(s') = t'$ for some $s' \in S$ and $t' \in T$ such that
$s \neq s'$ and $t \neq t'$.
We know $h$ is injective because $h(r) = g(f(r)) = g(s) = t$ and
$h(r') = g(f(r')) = g(s') = t'$ and $t \neq t'$.
\\ \\
We know $h$ is surjective because the Im $f$ is $S$ and Im $g$ is $T$.
Therefore, for all $t \in T$ there is a solution to $h(r) = t$.
\\ \\
We prove $h(r) + h(r') = h(r + r')$ as such:
\[
h(r) + h(r') = g(f(r)) + g(f(r')) = g(f(r + r')) = h(r + r')
\]
\\
We prove $h(r)h(r') = h(rr')$ as such:
\[
h(r)h(r') = g(f(r))g(f(r')) = g(f(r)f(r')) = g(f(rr')) = h(rr')
\]
Therefore $h$ is an isomorphism and $\cong$ is transitive.

\item Assume $\mathbb{C}$ and $\mathbb{R}$ are isomorphic and
$f: \mathbb{C} \rightarrow \mathbb{R}$.  By the definition of isomorphism,
we know that $f(1) = f(1\cdot1) = f(1)f(1) = f(1)^2$.  If $f(1) = f(1)^2$, then
$f(1)$ must either be 0 or 1.  Assume $f(1) = 0$.  Then, for an $c \in \mathbb{C}$,
$f(c) = f(c\cdot1) = f(c)f(1) = f(c)\cdot0 = 0$, and this is a contradiction since
isomorphisms are bijective.  Therefore, $f(1) = 1$.  Again, by the definition
of isomorphism, $f(-1) = -f(1) = -1$.  Now let $r = f(i)$.  As such,
$r^2 = f(i)f(i) = f(i^2) = f(-1) = -1$.  However, this is a contradiction
because there is no $r \in \mathbb{R}$ such $r^2 = -1$.  Therefore
$\mathbb{C}$ and $\mathbb{R}$ are not isomorphic.

\item Let $F$ denote a function $F: T_{[0,1]} \rightarrow T_{[0,2]}$ such that
$F(f)(x) = f(\frac{1}{2}x)$ for some $f \in T_{[0,1]}$.  If
$f_1, f_2 \in T_{[0,1]}$, we can prove $F(f_1 + f_2) = F(f_1) + F(f_2)$ as such:
\[
F(f_1+f_2)(x) = (f_1+f_2)(\frac{1}{2}x) = f_1(\frac{1}{2}x)+f_2(\frac{1}{2}x) = F(f_1)(x) + F(f_2)(x)
\]
We can prove $F(f_1 f_2) = F(f_1)F(f_2)$ as such:
\[
F(f_1 f_2)(x) = (f_1 f_2)(\frac{1}{2}x) = f_1(\frac{1}{2}x) f_2(\frac{1}{2})x = F(f_1)(x)F(f_2)(x)
\]
To prove injectivity, assume $F(f_1) = F(f_2)$:
\[
F(f_1)(x) = F(f_2)(x)
\] \[
f_1(\frac{1}{2}x) = f_2(\frac{1}{2}x)
\] \[
f_1(x) = f_2(x)
\]
Therefore $F(f_1) = F(f_2)$ implies $f_1 = f_2$. \\ \\
To prove surjectivity, let $h \in T_{[0,2]}$ and $f(x) = h(2x)$.
Therefore $F(f)(x) = f(\frac{1}{2}x) = h(x)$, implying that 
Im $F = T_{[0,2]}$ and $F$ is surjective. \\ \\
Thus, $T_{[0,1]}$ and $T_{[0,2]}$ are isomorphic.

\item Let $f: R \rightarrow \mathbb{Q}[\sqrt{2}]$ be the function
$f(\begin{bmatrix} a & b \\ 2b & a \end{bmatrix})$ = $a + b\sqrt2$ such that
$a, b \in \mathbb{Q}$. \\ \\

To prove injectivity, let $c, d \in \mathbb{Q}$ such that $a \neq c$ and $b \neq d$.
Also, assume $f(\begin{bmatrix} a & b \\ 2b & a \end{bmatrix}) = f(\begin{bmatrix} c & d \\ 2d & c \end{bmatrix})$.
As such:
\[
f(\begin{bmatrix} a & b \\ 2b & a \end{bmatrix}) = f(\begin{bmatrix} c & d \\ 2d & c \end{bmatrix})
\] \[
a + b\sqrt2 = c + d\sqrt2
\] \[
a - c = \sqrt{2}(d - b)
\] \[
\sqrt2 = (a - c)/(d - b)
\]
Since $\sqrt2$ is an irrational number and $(a-c)/(d-b)$ is a rational number,
this is a contradiction.  Thus, $f$ is injective. \\ \\
We know that $f$ is surjective because for all $a+b\sqrt2 \in \mathbb{Q}[\sqrt2]$,
there is an $\begin{bmatrix} a & b \\ 2b & a \end{bmatrix} \in M_2(\mathbb{R})$ such that
$f(\begin{bmatrix} a & b \\ 2b & a \end{bmatrix}) = a+b\sqrt2$. \\ \\

We can prove
$f(\begin{bmatrix} a & b \\ 2b & a \end{bmatrix}) + f(\begin{bmatrix} c & d \\ 2d & c \end{bmatrix})
= f(\begin{bmatrix} a & b \\ 2b & a \end{bmatrix} + \begin{bmatrix} c & d \\ 2d & c \end{bmatrix})$
 as such:
 \[
 f(\begin{bmatrix} a & b \\ 2b & a \end{bmatrix}) + f(\begin{bmatrix} c & d \\ 2d & c \end{bmatrix})
 \] \[
 = (a + bi) + (c+di) = (a+c) + (b+d)i
 \] \[
 = f(\begin{bmatrix} a+c & b+d \\ 2(b+d) & a+c \end{bmatrix})
 \] \[
 = f(\begin{bmatrix} a & b \\ 2b & a \end{bmatrix} + \begin{bmatrix} c & d \\ 2d & c \end{bmatrix})
 \]
 \\
 We can prove
 $f(\begin{bmatrix} a & b \\ 2b & a \end{bmatrix})f(\begin{bmatrix} c & d \\ 2d & c \end{bmatrix})
 = f(\begin{bmatrix} a & b \\ 2b & a \end{bmatrix}\begin{bmatrix} c & d \\ 2d & c \end{bmatrix})$
 as such:
 \[
f(\begin{bmatrix} a & b \\ 2b & a \end{bmatrix})f(\begin{bmatrix} c & d \\ 2d & c \end{bmatrix})
 \] \[
= (a+b\sqrt2)(c+d\sqrt2) = (ac + 2bd) + (ad + bc)\sqrt2
 \] \[
 = f(\begin{bmatrix} ac+2bd & ad+bc \\ 2(ad+bc) & ac + 2bd\end{bmatrix})
 = f(\begin{bmatrix} a & b \\ 2b & a \end{bmatrix}\begin{bmatrix} c & d \\ 2d & c \end{bmatrix})
 \]
 \\
 Therefore, $f$ is an isomorphism and $R$ is isomorphic to $\mathbb{Q}[\sqrt2]$.

\item The rings are not isomorphic because
$\mathbb{Z}/6\mathbb{Z} \times \mathbb{Z}/35\mathbb{Z}$
is a 170-element set and
$\mathbb{Z}/10\mathbb{Z} \times \mathbb{Z}/21\mathbb{Z}$
is an 180-element set and it is not possible to have a surjective
function from an 170-element set to an 180-element set.

\end{enumerate}
\end{document}
